\chapter*{Resumo}

A gênese e evolução de grandes bacias sedimentares no interior de continentes estáveis é um importante problema geológico que não é facilmente compreendido dentro do paradigma da Tectônica de Placas. A bacia do Parnaíba é uma das três grandes bacias sedimentares Paleozóicas na plataforma Sulamericana - junto com as bacias do Paraná e Amazonas. A bacia é comumente descrita como uma grande bacia cratônica do tipo sag com uma forma semi-circular e um depocentro que chega a profundidades de 3.5 km. Já o mecanismo físico que atuou na sua subsidência e evolução, em contrapartida, é muito controverso. Vários mecanismos de formação foram propostos nessa bacia, mas o conhecimento sobre a arquitetura crustal e mantélica abaixo da bacia do Parnaíba ainda é muito pequeno. Devido a isso, nós propomos investigar a arquitetura crustal abaixo da bacia do Parnaíba através da análise das Funções do Receptor e da tomografia de ruído ambiental, para imagear estruturas profundas e rasas, respectivamente. Na primeira parte do nosso trabalho, nós apresentamos estimativas pontuais de espessura crustal e razão Vp/Vs em 9 estações numa transecta NW-SE na parte central da bacia, junto com perfis de velocidade de onda S obtidos na inversão conjunta entre funções do receptor e velocidades de ondas de superfície. Essa análise revela a crosta como sendo espessa com 44-45 km em média no depocentro e um afinamento progressivo de 39-41 km em direção às bordas, enquanto que a razão Vp/Vs varia entre  1.70 e 1.78 ao longo da transecta, com grandes valores próximo ao depocentro. Os perfis de velocidade confirmam essa variação da espessura crustal e revelam uma crosta inferior de 18-22 km de profundidade com velocidades variando de 3.7 a 3.8 km/s, mas que podem chegar a 4.0-4.2 km/s perto do depocentro. Nossos resultados favorecem modelos de formação de bacia que possuem um estiramento crustal mínimo e são compatíveis com dobramentos flexurais devido a uma carga em profundidade. No entanto, não se observa a existência de um grande corpo intrusivo na crosta inferior. Arguimos que processos convectivos profundos na astenosfera podem prover tal mecanismo de carga. A segunda parte do nosso trabalho consistirá no imageamento da zona de transição do manto com funções de receptor para ver qual o comportamento da astenosfera abaixo da bacia. Além disso, utilizaremos o ruído ambiental para analisar as estruturas crustais rasas, principalmente para entender o efeito do lineamento Transbrasiliano na formação da bacia.