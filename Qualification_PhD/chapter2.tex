\chapter{Modelos de Formação de Bacias Intracratônicas}

\cite{allen_cratonic_2012} utiliza o termo bacia cratônica para com as bacias localizadas distantes de margens continentais convergentes e localizados numa variedade de substratos crustais, independentemente de serem escudos cristalinos, cinturões de dobramentos antigos e riftes complexos. Segundo \cite{armitage_subsidence_2010}, pode-se considerar que o interior dos continentes não sofre com a convecção do manto subjacente e com a tectônica de placas operante. No entanto, grandes regiões da litosfera continental estável tem experimentado ciclos de subsidência prolongada e de soerguimento, gerando grandes bacias. A principal feição das bacias cratônicas é o seu período de subsidência prolongada, o qual se iniciaria durante períodos de dispersão continental e podem continuar através de ciclos de fechamento de oceanos e colisão continental. Elas estão situadas no interior de margens passivas, mas são comumente conectadas ao oceano por riftes falhados que estão orientados em alto ângulo com a margem da placa, possíveis braços abortados de junções tríplices.

\cite{allen_cratonic_2012} mostra que apesar das individualidades de cada bacia, existem várias feições estruturais que podem ser comuns entre as bacias cratônicas, como: (1) a área da superfície ao redor da isópaca zero do preenchimento sedimentar é comumente circular ou elíptica; (2) em corte essas bacias possuem formato de pires, mostrando uma falta de falhas sin-tectônicas e uma espessura sedimentar menor que 5 km de profundidade; (3) a duração da subsidência é muito longa, por volta de centenas de milhões de anos, e as curvas de subsidência tectônica oriundas do backstripping apresentam uma tendência sublinear a um exponencial negativo suave. Essa longevidade da subsidência geralmente compreende várias fases de deposição da bacia separadas por inconformidades, gerando megasequências sobrepostas, que são produtos das mudanças na tectônica operante na época de deposição; (4) a estratigrafia é predominantemente de ambiente terrestre a marinho-raso, indicando que a sedimentação acompanhou o ritmo da subsidência tectônica. As bacias cratônicas em baixas paleolatitudes são geralmente dominadas por sedimentos quimiogênicos, mostrando que o suprimento de sedimento em partículas era modesto, e que altos topográficos estavam ausentes em torno das margens das bacias nesses locais; (5) bacias cratônicas são comumente espaçadas regularmente com seus centros localizados cerca de $10^{3}$ km de distância das bordas; (6) algumas bacias cratônicas estão associadas com o magmatismos amplamente difundidos, como a erupção de grandes volumes de basaltos. No entanto, as ligações entre o vulcanismo e o desenvolvimento da bacia ainda não são claras.  

Existem inúmeras tentativas de explicar a origem e o desenvolvimento de grandes bacias  sedimentares intracratônicas extensas e longevas, como pode ser visto nos trabalhos de \cite{hartley_interior_1994,kaminski_lithosphere_2000,armitage_subsidence_2010,mkenzie_speculations_2016}. O preenchimento sedimentar desse tipo de bacia raramente mostra-se falhado, sugerindo que deformação frágil raramente acompanha a subsiedência. \cite{hartley_interior_1994} propõem que a baixa a moderada maturidade dos indicadores termais nos sedimentos da bacia não está associada com as principais distorções termais da litosfera. Existiram dois períodos na história da Terra quando a iniciação de bacias cratônicas tem sido importantes, ambos relacionados com a quebra de assembléias supercontinentais. As bacias cratônicas antigas das quais se possuem mais dados são as pertencentes ao Craton norte-americano, que são associadas com a quebra do continente Laurência no começo do Paleozóico, como observado no trabalho de \cite{kaminski_lithosphere_2000}. 

\cite{hartley_interior_1994} e \cite{allen_cratonic_2012} enumeram as hipóteses propostas na literatura para explicar a criação e evolução das bacias intracratônicas, como (1) estiramento litosférico seguido de uma contração termal,  (2) mudanças de fase crustais e mantélicas,  metamorfismo e intrusão, (3) tensão planar ou rejuvenescimento tectônico, (4) instabilidades convectivas no manto, e (5) erosão subaérea de estruturas soerguidas. No entanto, uma melhor compreensão dos mecanismos de formação das bacias cratônicas é dificultada pela falta de conhecimento de algumas condições pretéritas das bacias. Além disso, o momento atual é um dispersão continental em vez de uma de ruptura supercontinental, como resultado, as bacias cratônicas podem ser características menos comuns da superfície da Terra hoje, quando comparadas a certos momentos no passado geológico, logo o observação de todos os fatores na formação e evolução deste tipo específico de bacia sedimentar é dificultada.




