\chapter*{Abstract}
The genesis and evolution of large basins in the stable interiors of continents is an important geological problem that is not easily understood within the Plate Tectonics paradigm. The Parnaíba basin is one of three large Paleozoic basins in stable South America - together with the Paraná and Amazon basin. The basin is commonly described as a large, sag-type cratonic basin, with a roughly circular shape and a depocenter reaching up to 3.5 km depth. The physical mechanism behind its subsidence and evolution, on the other hand, is more controversial. A number of basin-forming mechanisms have been proposed for this basin, but the knowledge about the deep architecture of this cratonic basin is very little yet. Owing to that, we propose to investigate the crustal and mantle architecture beneath of the Parnaíba basin by analyzing P-wave receiver functions and ambient noise tomography, to image deep and shallow structures, respectively. In the first part of our work, we present point estimates of crustal thickness and Vp/Vs ratio at 9 broadband stations in a 600 km-long transect crossing the central portion of the basin, along with detailed S-wave velocity-depth profiles obtained from the joint inversion of P-wave receiver functions and surface-wave dispersion velocities. This analysis reveals the crust could be as thick as 44-45 km around the basin’s depocenter and that it progressively thins to 39-41 km towards the edges, while bulk Vp/Vs ratios are in the 1.70-1.78 range along the transect, with larger values around the depocenter. The velocity-depth profiles confirm crustal thickness variations and reveal a lower crustal layer below 18-22 km depth, with S-velocities in the 3.7-3.8 km/s range that locally raise to 4.0-4.2 km/s near the depocenter. Our findings favor models invoking minimal stretching of the basin’s underlying crust and are found compatible with flexural bending by a deep load. However, the existence of a thick intrusive body pervading the lower crust is not supported by our results. We argue that deep convective processes in the asthenosphere might provide an alternative loading mechanism. The second part of our work will consists of imaging the transition zone of the mantle with P-wave receiver functions to see how is the behavior of the asthenosphere beneath the basin. Furthermore, to use the ambient noise to analyse the shallow crustal structures of the basin, mainly to understand the Transbrasiliano lineament effect in the basin evolution.
