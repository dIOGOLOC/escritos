\chapter{Resultados e Discussões}	

A caracterização do arcabouço estrutural utilizando métodos sismológicos possui problemas de unicidade de solução, como outros métodos geofísicos.  Essa falta de informação direta do objeto em estudo proporciona toda essa gam de soluções. Porém há inúmeros meios de se contornar tal situação. A modelagem ajuda na distinção das fases convertidas e das demais reverberações presentes na função do receptor.
Para delimitar as principais feições utilizou-se de modelos simples, apenas com camadas planas, com esses modelos pode-se mostrar que as estimativas da espessura crustal e a razão $V_{p}/V_{s}$ são consistentes com os dados observados.
