\chapter{Dispersão de Ondas de Superfície}

\section{Fundamentos do Método}

Os métodos para determinar a estrutura sísmica da Terra, em particular os métodos tomográficos, baseiam-se num princípio simples: a determinação das velocidades de propagação das ondas sísmicas e a procura de um modelo que melhor se ajuste às velocidades encontradas. A resolução dos modelos obtidos depende do tipo de onda utilizado e da geometria espacial das estações sismigráficas segundo à fonte do sinal. \cite{aki_space_1957} propôs a utilização do ruído sísmico ambiental para medir a dispersão das ondas Rayleigh e Love nas camadas mais superficiais. Somente \cite{campillo_long-range_2003}  e \cite{shapiro_emergence_2004} mostraram, pela primeira vez, a  presença de ondas superficiais nas correlações cruzadas entre pares de estações.

\cite{campillo_long-range_2003}, \cite{shapiro_emergence_2004} e, principalmente, \cite{wapenaar_retrieving_2004} mostram que pode-se recuperar a resposta elástica da Terra a partir da correlação cruzada entre dois pontos em  um campo de ondas difuso ou aleatório. Essa resposta é aproximada como a Função de Green, como é mostrada na equação \ref{crosscorrelation}. 

\cite{boschi_measuring_2013} define a correlação cruzada ($C_{xy}(t,\omega)$) como: 
\begin{eqnarray}
\label{crosscorrelation}
C_{xy}(t,\omega) = \frac{1}{2\pi}\int_{-T}^{T}u(x,t,\omega)u(y,t+\tau,\omega) d\tau
\end{eqnarray}

onde $u$ são sinais registrados em duas estações nas posições $x$ e $y$, $t$ é o tempo, $\omega$ é a frequência, $\tau$ é o atraso e o parâmetro T define o tamanho da janela que a correlação cruzada será computada.
\\

Por possuir inúmeras vantagens em relação aos métodos de análise tradicionais, o número de artigos analisando a dispersão de ondas de superfície cresceu bastante. \cite{shapiro_emergence_2004} lista as seguintes vantagens: as medidas podem ser realizadas em qualquer direção de propagação e não estão limitadas à geometria fonte-receptor; não dependem da localização da fonte; a zona de sensibilidade destas medições situa-se na região que fica entre as duas estações; pode-se analisar pequenos períodos se existirem estações relativamente próximas umas das outras.


\cite{shapiro_emergence_2004} testaram se as funções de Green podem ser extraídas do ruído sísmico ambiental. Neste teste eles selecionaram um
período relativamente calmo no nível de atividade sísmica mundial, onde não ocorreram sismos com magnitude menor que 7. Com esses registos contínuos da componente vertical das estações ANMO e CCM, vistas na Figura \ref{shapiro}-a). Com isso calcularam a correlação cruzada para diferentes bandas de período, mostrado na Figura \ref{shapiro}-b), e aplicaram a análise tempo/frequência para calcular a velocidade de grupo das ondas de superfície segundo \cite{keilis-borok_seismic_2013}. \cite{shapiro_emergence_2004}  compararam as características de dispersão do sinal emergente com mapas preditos para o mesmo trajeto.  Nesta comparação, verificaram que os resultados obtidos para os dois casos são semelhantes.

\begin{figure}[!ht]
\centering
\includegraphics[scale=0.7]{shapiro2004.png}
\caption{(a) Mapa mostrando a localização das estações. (b) Correlações cruzadas da componente vertical dos registros com diferentes filtros de passa-banda, indicados na parte esquerda superior. Linha pontilhada dá 
ênfase na dispersão do sinal emergente. Extraído de \cite{shapiro_emergence_2004}.}
\label{shapiro}
\end{figure} 

\section{Processamento}

Para o processamento dos dados utilizou-se o código escrito pelo Professor Bruno Goutorbe. Tal código engloba a preparação dos dados, o cálculo da correlação, a análise Tempo/frequência e também a inversão tomográfica.  O código está disponível no GitHub no seguinte repositório: https://github.com/bgoutorbe/seismic-noise-tomography.

Todo o fluxo de preparação e processamento dos dados  baseia-se no trabalho de  \cite{bensen_processing_2007}. Porém há alterações na filtragem espectral utilizada neste trabalho, devido a utilização de média a baixa frequência no processamento.


\begin{figure}[!ht]
\centering
\includegraphics[scale=0.7]{fluxograma_bensen2007.png}
\caption{Schematic representation of the data processing scheme. Phase 1 (described in Section 2 of the paper) shows the steps involved in preparing
single-station data prior to cross-correlation. Phase 2 (Section 3) outlines the cross-correlation procedure and stacking, Phase 3 (Section 4) includes dispersion measurement and Phase 4 (Section 5) is the error analysis and data selection process.. Adaptado de \cite{bensen_processing_2007}.}
\label{fluxograma_bensen2007}
\end{figure} 


