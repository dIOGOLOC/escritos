\chapter{Introdução}

A integração de dados geológicos e geofísicos multidisciplinares é a maneira de se conseguir galgar degraus no entendimento e na compreensão do arcabouço geológico complexo de nosso país. Para tal o Observatório Nacional juntamente com a Petrobras começaram uma parceria, e executaram o projeto “Imageamento Subsal pela Utilização Conjunta de Migração Pré-empilhamento em Profundidade, do Método Magnetotelúrico Marinho e do Método Gravimétrico” da Rede Temática de Estudos Geotectônicos da Petrobras. Esta integração de dados é vanguardista no Brasil, constituindo-se um modelo de estudos na exploração de hidrocarbonetos e também em estudos geotectônicos. O Observatório Nacional realizou levantamentos geofísicos na porção sudeste brasileira, especificamenteo entre a Faixa Brasília e a Faixa Ribeira. 

Esta dissertação apresentará os dados coletados de estações sismográficas instaladas neste projeto e os resultados gerados serão integrados com resultados de outros métodos geofísicos, e estes servirão como subsídios para uma melhor interpretação geológica da região. 

A área de estudo encontra-se sobre terrenos policíclicos referíveis ao sul do Cinturão de Dobramentos Ribeira, nomeada por \cite{Riccomini_1989}, Sul do Cráton São Francisco e Sul da Faixa Brasília, \cite{Almeida_Carneiro_1998}. Nessa região há um retrabalhamento de ciclos orogênicos pretéritos e o conjunto litológico é recortado por um sistema de falhas transcorrentes (zonas de cisalhamento) orientados segundo a estruturação regional, direção ENE a EW, \cite{Hasui_Sadowski_1976}. As feições estruturais da região de estudo são fortemente influenciadas pela Faixa Ribeira, devido a isso existe uma zona de interferência com a Faixa Brasília e com o Cráton do São Francisco,\cite{kuhn_metamorphic_2004}, \cite{heilbron_evolution_2010} e \cite{valeriano_u_pb_2011} e \cite{heilbron_serra_2013}.

A Faixa Ribeira já foi alvo de diversos estudos sismológicos para uma melhor entendimento das estruturas crustais. Autores como \cite{Bassini_1986}, \cite{souza_crustal_1991}, \cite{souza_shear-wave_1995}, \cite{assumpcao_crustal_2002},\cite{dias_cario_crustal_2006} e \cite{sand_franca_crustal_2004} já propuseram modelos de velocidade crustais para a região. Além disso existem inúmeros trabalhos geofísicos que também geraram conhecimento sobre a estruturas em subsuperfície. 

Para a análise da estrutura sísmica da Terra utilizou-se de dois métodos sismológicos que se baseiam-se num princípio simples: a determinação das velocidades de propagação das ondas sísmicas e a procura de um modelo que melhor se ajuste às velocidades encontradas. A resolução dos modelos obtidos depende do tipo de onda utilizado e da geometria espacial das estações sismográficas segundo à fonte do sinal. Para isso fez-se uso de dois métodos complementares, um para obter informações de regiões profundas da crosta, Função do Receptor, o outro foi utilizado para extrair informações mais rasas sobre a estrutura crustal, Correlação de Ruído Ambiental.

Para o cálculo a espessura crustal na região utilizou-se o método da Função do Receptor, que foi desenvolvido por \cite{clayton_source_1976}, \cite{Langston_1977}, \cite{ammon_isolation_1991}, \cite{cassidy_numerical_1992}, \cite{Zhu_Kanamori_2000}. Tal método faz uso do sinal de tele-sismos para inferir a profundidade da discontinuidade de Mohorovicic. Já para extrair informações sobre as estruturas crustais rasas utilizou-se as correlações cruzadas do ruído sísmico ambiental entre pares de estações para medir a dispersão das ondas Rayleigh nas camadas mais superficiais, inicialmente prosposto por \cite{aki_space_1957}, porém Somente \cite{campillo_long-range_2003}  e \cite{shapiro_emergence_2004} mostraram, pela primeira vez, a  presença de ondas superficiais nas correlações cruzadas.

O objetivo deste trabalho é a análise e delimitação de grandes feições estruturais crustais através de perfis de Funções do Receptor da onda P e de mapas tomográficos de velocidade gerados de estações sismográficas temporárias temporárias.