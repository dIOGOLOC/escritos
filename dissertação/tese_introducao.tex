\chapter{Introdução}

A integração de dados geológicos e geofísicos multidisciplinares é a maneira de se conseguir galgar degraus no entendimento e na compreensão do arcabouço geológico complexo de nosso país. Para tal o Observatório Nacional juntamente com a Petrobras começaram uma parceria, e executaram o projeto “Imageamento Subsal pela Utilização Conjunta de Migração Pré-empilhamento em Profundidade, do Método Magnetotelúrico Marinho e do Método Gravimétrico” da Rede Temática de Estudos Geotectônicos da Petrobras. Esta integração de dados é vanguardista no Brasil, constituindo-se um modelo de estudos na exploração de hidrocarbonetos e também em estudos geotectônicos. O Observatório Nacional realizou levantamentos geofísicos na porção sudeste brasileira, especificamenteo entre a Faixa Brasília e a Faixa Ribeira. 

A área de estudo encontra-se sobre terrenos policíclicos referíveis ao sul do Cinturão de Dobramentos Ribeira, nomeada por \cite{Riccomini_1989}, Sul do Cráton São Francisco e Sul da Faixa Brasília \citep{Almeida_Carneiro_1998}. Nessa região há um retrabalhamento de ciclos orogênicos pretéritos e o conjunto litológico é recortado por um sistema de falhas transcorrentes (zonas de cisalhamento) orientados segundo a estruturação regional, direção ENE a EW  \citep{Hasui_Sadowski_1976}. As feições estruturais da região de estudo são fortemente influenciadas pela Faixa Ribeira, devido a isso existe uma zona de interferência com a Faixa Brasília e com o Cráton do São Francisco \citep{kuhn_metamorphic_2004}; \citep{heilbron_evolution_2010}; \citep{valeriano_u_pb_2011}; \citep{heilbron_serra_2013}.

A Faixa Ribeira já foi alvo de diversos estudos sismológicos para um melhor entendimento das estruturas crustais. Autores como \cite{Bassini_1986}, \cite{souza_crustal_1991}, \cite{souza_shear-wave_1995}, \cite{assumpcao_crustal_2002}, \cite{dias_cario_crustal_2006} e \cite{sand_franca_crustal_2004} propuseram modelos de velocidade crustais para a região e fizeram considerações sobre a estruturação crustal, como a existência de camadas de baixa velocidade superficiais, discontinuidades de velocidade entre a Crosta superior e inferior, e até mesmo a presenção de uma inversão de velocidade sísmica. Além desses resultados apresentados sobre a região, outros métodos geofísicos foram aplicados no projeto SUBSAL, como os trabalhos de \cite{flora_solon_ancient_2013} e \cite{Silva_2014}, através do método magnetotelúrico e gravimétrico, respectivamente. Esses resultados corroboram com o contexto geológico prosposto anteriormente e mostram um nível de detalhe maior que outras pesquisas geofísicas realizadas na região. Estes trabalhos indicam que a crosta na região apresentam camadas superficiais de baixa velocidade e discontinuidades crustais intermediárias.

Os dados coletados de estações sismográficas instaladas no projeto SUBSAL e os resultados gerados nesta dissertação foram integrados com resultados \cite{flora_solon_ancient_2013} e \cite{Silva_2014} para ratificar a história geológica da região proposta na literatura. O objetivo deste trabalho é a análise e delimitação de grandes feições estruturais crustais através de perfis de Funções do Receptor da onda P e de mapas tomográficos de velocidade gerados a partir das estações temporárias do projeto SUBSAL mais algumas estações da Rede Sismográfica Brasileira (RSBR, \url{www.rsbr.gov.br}).Analisou-se a estrutura crustal da região através de dois métodos sismológicos diferentes, porém estes se baseiam na determinação das velocidades de propagação das ondas sísmicas e na busca de um modelo que melhor se ajuste às velocidades encontradas. A resolução dos modelos obtidos depende do tipo de onda utilizado e da geometria espacial das estações sismográficas segundo à fonte do sinal. Para isso fez-se uso de dois métodos complementares, um para se obter informações de regiões profundas da crosta, Função do Receptor, o outro foi utilizado para extrair informações mais rasas das estruturas crustais, Correlação de Ruído Sísmico Ambiental.

Para o cálculo a espessura crustal na região utilizou-se o método da Função do Receptor, que foi desenvolvido por \cite{clayton_source_1976}, \cite{Langston_1977}, \cite{ammon_isolation_1991}, \cite{cassidy_numerical_1992}, \cite{Zhu_Kanamori_2000}. Tal método faz uso do sinal de telessismos para inferir a profundidade da discontinuidade de Mohorovicic. Já para extrair informações sobre as estruturas crustais rasas utilizou-se as correlações cruzadas do ruído sísmico ambiental entre pares de estações para medir a dispersão das ondas Rayleigh nas camadas mais superficiais, inicialmente prosposto por \cite{aki_space_1957}, porém, somente \cite{campillo_long-range_2003}  e \cite{shapiro_emergence_2004} mostraram, pela primeira vez, a  presença de ondas superficiais nas correlações cruzadas de ruído sísmico.

A discontinuidade de Moho estimada é maior no interior do continente do que na parte costeira. As estações que bordeiam o Cráton do São Francisco e a Bacia do Paraná tem uma espessura média de Moho de 40 km. Já nas outras estações mostram uma espessura de Moho crescente quanto mais próximas da Faixa Brasília, Cráton do São Francisco e da Bacia do Paraná. Os perfis exemplificados nesta dissertação indicam o afinamento crustal em direção ao oceano e uma heterogeneidade crustal, tal crosta pode conter camadas de baixa velocidade na superfície, uma inversão de velocidade e uma discontinuidade intermediária. Os resultados da Tomografia Sísmica de Ruído Ambiental delimitam as grandes feições estruturais da região, como a Bacia do Paraná, Bacia de Taubaté, Faixa Brasília, principalmente no mapa tomográfico com período de 5 segundos.

Para reforçar os resultados obtidos é necessário um estudo detalhado sobre a heterogeneidade lateral da crosta com as Funções do Receptor, além de uma inversão conjunta em profundidade das Funções do Receptor e da Correlação de Ruído Ambiental para observar as estruturas crustais em profundidade.