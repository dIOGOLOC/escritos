\chapter{Contexto Geológico}
\date{24}{03}{2015}


A área de estudo enquadra-se geolologicamente no Rift Continental do Sudeste do Brasil(RCSB) sobre terrenos policíclicos referíveis ao sul do Cinturão de Dobramentos Ribeira, nomeada por \cite{Riccomini_1989} em seu trabalho, Sul do Cráton São Francisco e Sul da Faixa Brasília, como pode ser observado na Figura \ref{mapa_geologico}. Essa zona geológica é intulada por \cite{Almeida_Carneiro_1998} como Planalto Atlântico. Encontra-se nessa região retrabalhamento de ciclos orogênicos pretéritos e o conjunto lito1ógico é recortado por um sistema de falhas transcorrentes (zonas de cisalhamento) orientados segundo a estruturação regional, direção ENE a EW, \cite{Hasui_Sadowski_1976}. As feições estruturais da região de estudo são fortemente influenciadas pelo Cinturão de Dobramentos Ribeira.

As bacias sedimentares presentes na área de estudo estão alojadas na região do Rift Continental do Sudeste do Brasil (RCSB), como pode ser visto na Figura \ref{mapa_estacoes_geologico}. \cite{Riccomini_1989} apresenta o RCSB como uma depressão alongada e deprimida com mais de 900 km de comprimento entre os estados Paraná e Rio de Janeiro. Este Rift possui uma idade paleógena e segue a linha de costa atual, alcançando o Oceano Atlântico em seu segmento ocidental e na sua Terminação nordeste. Inúmeros corpos alcalinos de idade cretácica a paleogênica ocorrem ao longo das bordas desse sistema de Rifts. A área em estudo engloba o segmento central do RCSB. Este possui as bacias sedimentares de São Paulo, Taubaté, Resende e Volta Redonda, como pode-se observar nas áreas com tonalidades em amarelo na Figura \ref{mapa_estacoes_geologico}. 

É visível a presença de corpos arredondas nas Figuras \ref{mapa_geologico} e \ref{mapa_estacoes_geologico}, tais corpos representam plútons alcalinos cretácicos e cenozóicos. \cite{MOTA_2012} mostra que as intrusões alcalinas geram grandes desníveis topográficos, áreas elevadas que podem atingir 800 metros acima do nível do mar. Essas rochas estão alinhadas na direção WSW-ENE, como pode ser visto na Figura \ref{mapa_geologico}. A maior parte desses corpos magmáticos é formada por sienitos e monzonitos, com variações texturais desde plutônicas a subvulcânicas.

\begin{figure}[!ht]
\centering
\includegraphics[scale=0.5]{Figs/mapa_geologico.png}
\caption[Mapa tectônico da Região do Sudeste do Brasil segundo  \cite{trouw_new_2013}.]{Mapa tectônico da Região do Sudeste do Brasil com a área de trabalho marcada pelo quadradro. Legenda: 1-Bacias do Paraná e do Rift Cenozóico. 2-Plutons alcalinos Cenozóicos/Cretácios. Cráton São Francisco e Bacias interiores (3–5), 3-Embasamento; 4-Cobertura (Grupo Bambuí); 5-Cobertura (rochas metasedimentares autóctones e paraautóctones. Orógeno Brasília (6-9) 6-Sistema de \textit{Nappe} Andrelândia(AND) e \textit{Nappe} Passos(P);7-\textit{Nappe} Socorro(S)-Guaxupé(G); 8- Terreno Embu(E)-Paraíba do Sul(PS); 9-Terreno Apiá. Orógeno Ribeira(6-14), 10-Domínio Externo; 11- Domínio Juiz de Fora; 12-Arco Rio Negro(Terreno Oriental); 13-Terreno Oriental; 14- Terreno Cabo Frio. A área demarcada com a linha tracejada cobrindo a parte sul da Faixa Brasília e a parte sudeste do Cráton São Francisco corresponde a uma zona de interferência onde a deformação e o metamorfismo da Faixa Ribeira se  sobressai na Faixa Brasília. Adaptado de \cite{trouw_new_2013}}
\label{mapa_geologico}
\end{figure} 

O sul da Faixa Brasilía foi descrito, \cite{pimentel_tectonic_2011},\cite{reno_situ_2012} e \cite{trouw_new_2013}, como resultado da colisão entre a margem passiva do paleocontinente São Francisco do leste com a margem ativa do bloco, ou paleocontinente, Paranapanema do lado oeste da sutura, observado no perfil A-B na Figura \ref{perfil_esquematico}. Esta colisão produziu um empilhamento espesso de \textit{nappes} ao longo da sutura, como o Sistema de \textit{Nappe} Andrelândia(ANS), que pode ser observado nos perfis mostrados no perfil A-B mostrado na Figura \ref{perfil_esquematico}. Dobras em bainha em grande escala e inúmeras dobras interropidas atestam a deformação dúctil intensa dentro das \textit{Nappes}. Lineamentos alongado, combinados com indicadores de cisalhamento mostram do norte para o sul a troca progressiva, cavalgando do topo para E-SE  na \textit{Nappe} Passos. A sutura desse cinturão é interpretada sendo localizada entre a \textit{Nappe} Socorro-Guaxupé e o Sistema \textit{Nappe} Andrelândia, como mostrado no perfil C-D na Figura \ref{perfil_esquematico}.

\begin{figure}[!ht]
\centering
\includegraphics[scale=0.5]{Figs/perfil_esquematico_area.png}
\caption[Perfis Esquemáticos da Região do Sudeste do Brasil segundo  \cite{trouw_new_2013}.]{Perfis Esquemáticos marcados na Figura \ref{mapa_geologico} mostrando a evolução da superposição na zona de interferência. Retirado de \cite{trouw_new_2013}}
\label{perfil_esquematico}
\end{figure} 

A Faixa Ribeira é composta por rochas metamórficas, migmatitos e granitóides relacionados ao Ciclo Orogenético Brasiliano, como citam \cite{kuhn_metamorphic_2004}, \cite{heilbron_evolution_2010} e \cite{valeriano_u_pb_2011}. Esta tendência estrutural regional NE-SW pode ser observada na Figura \ref{mapa_geologico}. Segundo \cite{heilbron_evolution_2010} a Faixa Ribeira é composta por 4 terrenos tectônicos-estratigráficos separados por falhas de
empurrão ou por zonas de cisalhamento oblíquas transpressivas: (a) a margem retrabalhada do Cráton São Francisco definida como Terreno Ocidental; (b) O Terreno Paraíba do Sul-Embú que está cavalgando sobre o Terreno Ocidental;(c) O Terreno Oriental (Serra do Mar) que inclui o Arco Magmático Neoproterozóico, e (d) O Terreno Cabo Frio, que foi acrescionado depois, por volta de 520 M.a. Estes Terrenos estão demarcados na Figura \ref{mapa_geologico}. Estes são subdivididos em vários domínios, tais domínios são identificados devido ao seu contraste litológico, geoquímica isotópica e geocronologia, cita \cite{kuhn_metamorphic_2004}. A sutura entre o Terreno Ocidental e Oriental é uma zona de cisalhamento mergulhando para noroeste (NW), também chamada de Limite Tectônico Central, \textit{Central Tectonic Boundary}(CTB) por \cite{heilbron_evolution_2010} e \cite{trouw_new_2013}, demarcada na Figura \ref{mapa_geologico}. Essa sutura pode ser mapeada continuamente por pelo menos 200 km entre a costa de São Paulo e a Serra do Órgãos, no estado do Rio de Janeiro. 

\cite{trouw_new_2013} sumariza as principais características dos terrenos que compõem a Faixa Ribera. O Terreno Ocidental é caraterizado por rochas do embasamento Paleoproterozóico a Arqueano, representado pelos Complexos Barbacena, Mantiqueira e Juiz de Fora, e por uma cobertura siliciclástica metamorfizada oriunda de uma margem passiva. Esta é chamada de Megassequência Deposicional Andrelândia. Já o Terreno ou Klippe Paraíba do Sul possui ortognaisses granodioríticos a graníticos do Complexo Quirino, de idade Paleoproterozóica e é coberta por rochas do Complexo Paraíba do Sul, gnaisses feldspáticos e pelíticos, com intercalações de mármores dolomíticos. O Terreno Oriental aloja ortognaisses tonalíticos a granodioríticos pertencentes ao Complexo Rio Negro, bem como metassedimentos Neoproterozóicos, ricos em intercalações carbonáticas e rochas metabásicas. A Colisão deste terrenos acarretou na geração de vários tipos de rochas granitóides sin-colisionais: leucogranitos, charnockitos, granitos porfiróides e bitotita granitos. Por fim, o Terreno Cabo Frio engloba ortognaisses Paleoproterozóicos do Complexo Quirino e uma sucessão metassedimentar com gnaisses pelíticos com cianita, sillimanita e granada, metabasitos e rochas calcissilicáticas. 

\begin{figure}[!ht]
\centering
\includegraphics[scale=0.5]{Figs/mapa_estacoes_geologico.png}
\caption[Mapa de localização das Estações Sismográficas na região com o mapa geológico simplificado.]{Mapa de localização das Estações Sismográficas utilizadas neste trabalho e o mapa simplificado das unidades estratigráficas e tectônicas. BP-Bacia do Paraná, CSF-Cráton São Francisco, FB-Faixa Brasília, FR-Faixa Ribeira, TO-Terreno Oriental.}
\label{mapa_estacoes_geologico}
\end{figure}

