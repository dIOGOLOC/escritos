\chapter{Conclusões e Cenários Futuros}

Nesta dissertação foi utilizado duas fontes sísmicas diferentes para se analisar a estruturação da crosta na Faixa Ribeira. Nas Funções do Receptor fez-se uso de sismos para se recuperar a estrutura abaixo das estações sismográficas. Já as Correlações de Ruído Sísmico Ambiental investigou-se a estrutura da crosta nos trajetos entre as estações. Estes dois métodos foram complementares para uma análise tanto rasa quanto profunda da crosta terrestre.

Os resultados encontrados pelas Funções do Receptor geram resultados consistentes do limite crustal na região de estudo. A discontinuidade de Moho estimada é maior no interior do continente do que na parte costeira, corroborando com os resultados estimados por \cite{sand_franca_crustal_2004}, \cite{Assumpcao_America_2013}, \cite{Assumpcao_Brazil_2013} e \cite{van_der_meijde_gravity_2013}. As estações que bordeiam o Cráton do São Francisco e a Bacia do Paraná tem uma espessura média de Moho de 40 km. Já nas outras estações mostram uma espessura de Moho crescente quanto mais próximas da Faixa Brasília, Cráton do São Francisco e da Bacia do Paraná. Os perfis exemplificados nesta dissertação indicam o afinamento crustal em direção ao oceano e uma heterogeneidade crustal, tal crosta pode conter camadas de baixa velocidade na superfície, uma inversão de velocidade e uma discontinuidade intermediária.

Os resultados da Tomografia Sísmica de Ruído Ambiental delimitam as grandes feições estruturais da região, como a Bacia do Paraná, Bacia de Taubaté, Faixa Brasília, principalmente no mapa tomográfico com período de 5 segundos. A Faixa Brasília apresentou uma inversão de velocidade que é condizente com inúmeros modelos de velocidade apresentados para a região, como os modelos apresentados por \cite{sand_franca_crustal_2004}, \cite{dias_cario_crustal_2006}, \cite{flora_solon_ancient_2013}, \cite{Silva_2014}. Observa-se nas Correlações cruzadas a existência de uma influência de fontes direcionais nos registros. 

Os resultados gerados nesta dissertação integrados com os resultados de outros métodos geofísicos do projeto SUBSAL, realizados por \cite{flora_solon_ancient_2013} e \cite{Silva_2014}, conseguem ilustrar melhor o arcabouço geológico da região. Tais resultados ajudam na delimitação dos limites estruturais em profundidade na região da Faixa Ribeira, principalmente para os sistemas de \textit{Nappes} na região, estes citados por  \cite{heilbron_evolution_2010}, \cite{valeriano_u_pb_2011},  \cite{heilbron_serra_2013} e \cite{trouw_new_2013}. 

Para complementar e reforçar os resultados obtidos nesta dissertação propusemos estudos detalhados sobre a heterogeneidade lateral da crosta através da amplitude das componentes radial e transversal das Funções do Receptor e da análise das fontes de ruído sísmico ambiental da região sudeste. Além disso, uma análise tempo/frequência das correlações cruzadas com a frequência instantânea ao invés da frequência central se faz necessária para dimuinuir o vazamento espectral. Pretende-se fazer uma inversão conjunta das Funções do Receptor e da Tomografia Sísmica de Ruído Ambiental para apresentar resultados mais consistentes sobre a estrutura crustal local.