\chapter{Conclusões e Cenários Futuros}

Nesta dissertação utilizou-se de duas fontes diferentes para se analisar a estruturação da crosta na Faixa Ribeira. Nas Funções do Receptor fez-se uso de sismos para se recuperar a estrutura abaixo das estações sismográficas. Já as Correlações de Ruído Sísmico Ambiental registrado em 38 estações, tanto permanentes quanto temporárias, ao longo de aproximadamente duas anos nas estações temporárias e de 1995 a 2012 nas estações permanentes, com o objetivo de investigar a estrutura da crosta nos trajetos entre as estações. Com estes registos foi possível obter funções de Green que permitiram obter as curvas de dispersão das ondas de Rayleigh que gerou mapas tomográficos sísmicos da região. 

Os resultados encontrados pelas Funções do Receptor geram resultados consistentes sobre o limite da crosta na região. A discontinuidade de Moho estimada é maior no interior do continente do que na parte costeira, corroborando com os resultados pretéritos de \cite{sand_franca_crustal_2004}, \cite{Assumpcao_America_2013}, \cite{Assumpcao_Brazil_2013} e \cite{van_der_meijde_gravity_2013}. As estações que bordeiam o Cráton do São Francisco e a Bacia do Paraná tem uma espessura média de Moho de 40 km. Já nas outras estações tendem a uma espessura de Moho crescente quando mais próximas da Faixa Brasília, Cráton do São Francisco e da Bacia do Paraná. Corroborado com os perfis mostrados que indicam o afinamento crustal em direção ao oceano.

Os resultados da Tomografia Sísmica de Ruído Ambiental delimitam as grandes feições estruturais da região, como a Bacia do Paraná, Bacia de Taubaté, Faixa Brasília, principalmente no mapa com período de 5 segundos. A Faixa Brasília apresentou uma inversão de velocidade que é condizente com inúmeros modelos apresentados na literatura, como \cite{dias_cario_crustal_2006}, \cite{sand_franca_crustal_2004}, \cite{flora_solon_ancient_2013}, \cite{Silva_2014}. Observa-se nas Correlações cruzadas a existência de uma influência de fontes direcionais nos registros. 

Os resultados gerados nesta dissertação integrados com os resultados de outros métodos geofísicos do projeto SUBSAL, realizados por \cite{flora_solon_ancient_2013} e \cite{Silva_2014}, conseguem ilustrar melhor o arcabouço geológico da região. Tais resultados ajudam na delimitação dos limites estruturais em profundidade na região a Faixa Ribeira. 

As perspectivas futuras para complementar esse trabalho e reforçar os resultados obtidos estão em estudos mais detalhados sobre a heterogeneidade lateral da crosta através Funções do Receptor e sobre a análise das fontes de ruído sísmico ambiental. Realizar uma análise tempo/frequência das correlações cruzadas com a frequência instantânea e uma inversão conjunta dos dados para de Funções do Receptor e de Tomografia Sísmica de Ruído Ambiental.