\chapter*{Abstract}
\addcontentsline{toc}{chapter}{Abstract}

Twenty-four broadband temporary seismographic stations were installed along three profiles, two perpendicular to the coast and one parallel, the interstation distance is approximately 20 km. The geological context of the study area includes the Southeast of Brazil Continental Rift, covered by polycyclic terrains from the south of the Ribeira Belt. The main goal is determine the crustal structure and highlight the velocity discontinuities in order to better understand the regional geological framework. An effort has been made to ensure the data quality in temporary stations. Tests with the P wave arrival time and noise level help to localize some errors in database. The receiver function method has been applied to observe the inner crustal discontinuities bellow each station and, posteriorly, to compute the Moho depth. The results of the Receiver Functions indicate that Moho decreases from the inner of continent to coastal area, from 40 km to 33 km depth. Rayleigh wave group velocities are measured by the cross-correlation of ambient noise in the temporary stations. The group velocities was used to determine the shallow crustal structures. The maps of Rayleigh wave group velocity are consistent with the shallow crustal structure described in the literature. The study highlights the advantages of using several methods to characterize the crustal framework. 