\chapter*{Resumo}	
\addcontentsline{toc}{chapter}{Resumo}

O contexto geológico da área em estudo inclui o Rift Continental do Sudeste do Brasil, coberto por terrenos policíclicos do sul da Faixa Ribeira. Vinte e quatro estações sismográficas temporárias de banda larga foram instaladas ao longo de três perfis, dois perpendiculares a costa e um paralelo, a distância entre as estações é de $\simeq 20 km$.  O objetivo principal é determinar a estrutura da crosta e destacar as descontinuidades de velocidade sísmica, a fim de melhor compreender o arcabouço geológico regional. Há um esforço para garantir a qualidade dos dados nas estações temporárias. Os testes com o tempo de chegada da onda P e com o nível de ruído ajudaram a localizar alguns erros no banco de dados. O método da Função de Receptor foi aplicado para observar as descontinuidades internas da crosta abaixo de cada estação e, posteriormente, para calcular a profundidade de Moho. Os resultados das Funções do Receptor indicam que Moho diminui a partir do interior do continente para a região costeira, de 40 km para 33 km de profundidade. A velocidade de grupo das ondas Rayleigh são medidas pela correlação cruzada do Ruído Ambiental nas estações temporárias e foi utilizada para determinar as estruturas superficiais da crosta terrestre. Os mapas de velocidade de grupo das ondas Rayleigh são consistentes com a estrutura superficial da crosta terrestre descrita na literatura. Esta dissertação destaca as vantagens da utilização de vários métodos para caracterizar a estrutura da crosta terrestre.